% Options for packages loaded elsewhere
\PassOptionsToPackage{unicode}{hyperref}
\PassOptionsToPackage{hyphens}{url}
%
\documentclass[
]{article}
\usepackage{lmodern}
\usepackage{amssymb,amsmath}
\usepackage{ifxetex,ifluatex}
\ifnum 0\ifxetex 1\fi\ifluatex 1\fi=0 % if pdftex
  \usepackage[T1]{fontenc}
  \usepackage[utf8]{inputenc}
  \usepackage{textcomp} % provide euro and other symbols
\else % if luatex or xetex
  \usepackage{unicode-math}
  \defaultfontfeatures{Scale=MatchLowercase}
  \defaultfontfeatures[\rmfamily]{Ligatures=TeX,Scale=1}
\fi
% Use upquote if available, for straight quotes in verbatim environments
\IfFileExists{upquote.sty}{\usepackage{upquote}}{}
\IfFileExists{microtype.sty}{% use microtype if available
  \usepackage[]{microtype}
  \UseMicrotypeSet[protrusion]{basicmath} % disable protrusion for tt fonts
}{}
\makeatletter
\@ifundefined{KOMAClassName}{% if non-KOMA class
  \IfFileExists{parskip.sty}{%
    \usepackage{parskip}
  }{% else
    \setlength{\parindent}{0pt}
    \setlength{\parskip}{6pt plus 2pt minus 1pt}}
}{% if KOMA class
  \KOMAoptions{parskip=half}}
\makeatother
\usepackage{xcolor}
\IfFileExists{xurl.sty}{\usepackage{xurl}}{} % add URL line breaks if available
\IfFileExists{bookmark.sty}{\usepackage{bookmark}}{\usepackage{hyperref}}
\hypersetup{
  pdftitle={Student Crimtab Data Graphic Analysis},
  pdfauthor={coop711},
  hidelinks,
  pdfcreator={LaTeX via pandoc}}
\urlstyle{same} % disable monospaced font for URLs
\usepackage[margin=1in]{geometry}
\usepackage{color}
\usepackage{fancyvrb}
\newcommand{\VerbBar}{|}
\newcommand{\VERB}{\Verb[commandchars=\\\{\}]}
\DefineVerbatimEnvironment{Highlighting}{Verbatim}{commandchars=\\\{\}}
% Add ',fontsize=\small' for more characters per line
\usepackage{framed}
\definecolor{shadecolor}{RGB}{248,248,248}
\newenvironment{Shaded}{\begin{snugshade}}{\end{snugshade}}
\newcommand{\AlertTok}[1]{\textcolor[rgb]{0.94,0.16,0.16}{#1}}
\newcommand{\AnnotationTok}[1]{\textcolor[rgb]{0.56,0.35,0.01}{\textbf{\textit{#1}}}}
\newcommand{\AttributeTok}[1]{\textcolor[rgb]{0.77,0.63,0.00}{#1}}
\newcommand{\BaseNTok}[1]{\textcolor[rgb]{0.00,0.00,0.81}{#1}}
\newcommand{\BuiltInTok}[1]{#1}
\newcommand{\CharTok}[1]{\textcolor[rgb]{0.31,0.60,0.02}{#1}}
\newcommand{\CommentTok}[1]{\textcolor[rgb]{0.56,0.35,0.01}{\textit{#1}}}
\newcommand{\CommentVarTok}[1]{\textcolor[rgb]{0.56,0.35,0.01}{\textbf{\textit{#1}}}}
\newcommand{\ConstantTok}[1]{\textcolor[rgb]{0.00,0.00,0.00}{#1}}
\newcommand{\ControlFlowTok}[1]{\textcolor[rgb]{0.13,0.29,0.53}{\textbf{#1}}}
\newcommand{\DataTypeTok}[1]{\textcolor[rgb]{0.13,0.29,0.53}{#1}}
\newcommand{\DecValTok}[1]{\textcolor[rgb]{0.00,0.00,0.81}{#1}}
\newcommand{\DocumentationTok}[1]{\textcolor[rgb]{0.56,0.35,0.01}{\textbf{\textit{#1}}}}
\newcommand{\ErrorTok}[1]{\textcolor[rgb]{0.64,0.00,0.00}{\textbf{#1}}}
\newcommand{\ExtensionTok}[1]{#1}
\newcommand{\FloatTok}[1]{\textcolor[rgb]{0.00,0.00,0.81}{#1}}
\newcommand{\FunctionTok}[1]{\textcolor[rgb]{0.00,0.00,0.00}{#1}}
\newcommand{\ImportTok}[1]{#1}
\newcommand{\InformationTok}[1]{\textcolor[rgb]{0.56,0.35,0.01}{\textbf{\textit{#1}}}}
\newcommand{\KeywordTok}[1]{\textcolor[rgb]{0.13,0.29,0.53}{\textbf{#1}}}
\newcommand{\NormalTok}[1]{#1}
\newcommand{\OperatorTok}[1]{\textcolor[rgb]{0.81,0.36,0.00}{\textbf{#1}}}
\newcommand{\OtherTok}[1]{\textcolor[rgb]{0.56,0.35,0.01}{#1}}
\newcommand{\PreprocessorTok}[1]{\textcolor[rgb]{0.56,0.35,0.01}{\textit{#1}}}
\newcommand{\RegionMarkerTok}[1]{#1}
\newcommand{\SpecialCharTok}[1]{\textcolor[rgb]{0.00,0.00,0.00}{#1}}
\newcommand{\SpecialStringTok}[1]{\textcolor[rgb]{0.31,0.60,0.02}{#1}}
\newcommand{\StringTok}[1]{\textcolor[rgb]{0.31,0.60,0.02}{#1}}
\newcommand{\VariableTok}[1]{\textcolor[rgb]{0.00,0.00,0.00}{#1}}
\newcommand{\VerbatimStringTok}[1]{\textcolor[rgb]{0.31,0.60,0.02}{#1}}
\newcommand{\WarningTok}[1]{\textcolor[rgb]{0.56,0.35,0.01}{\textbf{\textit{#1}}}}
\usepackage{graphicx,grffile}
\makeatletter
\def\maxwidth{\ifdim\Gin@nat@width>\linewidth\linewidth\else\Gin@nat@width\fi}
\def\maxheight{\ifdim\Gin@nat@height>\textheight\textheight\else\Gin@nat@height\fi}
\makeatother
% Scale images if necessary, so that they will not overflow the page
% margins by default, and it is still possible to overwrite the defaults
% using explicit options in \includegraphics[width, height, ...]{}
\setkeys{Gin}{width=\maxwidth,height=\maxheight,keepaspectratio}
% Set default figure placement to htbp
\makeatletter
\def\fps@figure{htbp}
\makeatother
\setlength{\emergencystretch}{3em} % prevent overfull lines
\providecommand{\tightlist}{%
  \setlength{\itemsep}{0pt}\setlength{\parskip}{0pt}}
\setcounter{secnumdepth}{-\maxdimen} % remove section numbering

\title{Student Crimtab Data Graphic Analysis}
\author{coop711}
\date{2021-02-10}

\begin{document}
\maketitle

\hypertarget{data-manipulation}{%
\subsection{Data Manipulation}\label{data-manipulation}}

\begin{Shaded}
\begin{Highlighting}[]
\KeywordTok{load}\NormalTok{(}\StringTok{"./crimtab.RData"}\NormalTok{)}
\end{Highlighting}
\end{Shaded}

산점도를 여러 유형으로 표현하기 위하여 필요한 패키지 설치

\begin{Shaded}
\begin{Highlighting}[]
\CommentTok{# install.packages("hexbin", repos = "https://cran.rstudio.com")}
\KeywordTok{library}\NormalTok{(hexbin)}
\end{Highlighting}
\end{Shaded}

\texttt{crimtab\_bin} 계산

\begin{Shaded}
\begin{Highlighting}[]
\NormalTok{crimtab_bin <-}\StringTok{ }\KeywordTok{hexbin}\NormalTok{(crimtab_long_df}\OperatorTok{$}\NormalTok{height, }
\NormalTok{                      crimtab_long_df}\OperatorTok{$}\NormalTok{finger, }
                      \DataTypeTok{xbins =} \DecValTok{50}\NormalTok{)}
\end{Highlighting}
\end{Shaded}

\hypertarget{plots}{%
\subsection{Plots}\label{plots}}

\begin{Shaded}
\begin{Highlighting}[]
\KeywordTok{par}\NormalTok{(}\DataTypeTok{mfrow =} \KeywordTok{c}\NormalTok{(}\DecValTok{2}\NormalTok{, }\DecValTok{2}\NormalTok{))}
\CommentTok{# plot(x = crimtab_long_df[, 2], y = crimtab_long_df[, "finger"])}
\KeywordTok{plot}\NormalTok{(crimtab_long_df[, }\DecValTok{2}\OperatorTok{:}\DecValTok{1}\NormalTok{])}
\KeywordTok{plot}\NormalTok{(crimtab_long_df[, }\DecValTok{2}\OperatorTok{:}\DecValTok{1}\NormalTok{], }
     \DataTypeTok{pch =} \DecValTok{20}\NormalTok{)}
\CommentTok{# smoothScatter(crimtab_long_df[,"height"], crimtab_long_df[,"finger"], xlab = "height", ylab = "finger")}
\CommentTok{# smoothScatter(crimtab_long_df[,"height"], crimtab_long_df[,"finger"], nbin = 32, xlab = "height", ylab = "finger")}
\KeywordTok{smoothScatter}\NormalTok{(crimtab_long_df, }
              \DataTypeTok{xlab =} \StringTok{"height"}\NormalTok{, }
              \DataTypeTok{ylab =} \StringTok{"finger"}\NormalTok{)}
\KeywordTok{smoothScatter}\NormalTok{(crimtab_long_df, }
              \DataTypeTok{nbin =} \DecValTok{32}\NormalTok{, }
              \DataTypeTok{xlab =} \StringTok{"height"}\NormalTok{, }
              \DataTypeTok{ylab =} \StringTok{"finger"}\NormalTok{)}
\end{Highlighting}
\end{Shaded}

\includegraphics{crimtab_graphics_files/figure-latex/plots-1.pdf}

\hypertarget{plot-crimtab_bin}{%
\subsubsection{Plot crimtab\_bin}\label{plot-crimtab_bin}}

\begin{Shaded}
\begin{Highlighting}[]
\KeywordTok{par}\NormalTok{(}\DataTypeTok{mfrow =} \KeywordTok{c}\NormalTok{(}\DecValTok{1}\NormalTok{, }\DecValTok{1}\NormalTok{))}
\KeywordTok{plot}\NormalTok{(crimtab_bin, }
     \DataTypeTok{xlab =} \StringTok{"height(inches)"}\NormalTok{, }
     \DataTypeTok{ylab =} \StringTok{"finger length(cm)"}\NormalTok{)}
\end{Highlighting}
\end{Shaded}

\includegraphics{crimtab_graphics_files/figure-latex/plot crimtab_bin-1.pdf}

\hypertarget{uxc0b0uxc810uxb3c4uxc640-uxd568uxaed8-uxc8fcuxbcc0uxbd84uxd3ec-uxd45cuxc2dc}{%
\subsubsection{산점도와 함께 주변분포
표시}\label{uxc0b0uxc810uxb3c4uxc640-uxd568uxaed8-uxc8fcuxbcc0uxbd84uxd3ec-uxd45cuxc2dc}}

\begin{Shaded}
\begin{Highlighting}[]
\KeywordTok{par}\NormalTok{(}\DataTypeTok{mar =} \KeywordTok{c}\NormalTok{(}\DecValTok{4}\NormalTok{, }\DecValTok{4}\NormalTok{, }\DecValTok{1}\NormalTok{, }\DecValTok{1}\NormalTok{))}
\KeywordTok{par}\NormalTok{(}\DataTypeTok{fig =} \KeywordTok{c}\NormalTok{(}\DecValTok{0}\NormalTok{, }\FloatTok{0.8}\NormalTok{, }\DecValTok{0}\NormalTok{, }\FloatTok{0.8}\NormalTok{))}
\KeywordTok{plot}\NormalTok{(crimtab_long_df, }\DataTypeTok{pch =} \DecValTok{20}\NormalTok{)}
\KeywordTok{par}\NormalTok{(}\DataTypeTok{fig =} \KeywordTok{c}\NormalTok{(}\DecValTok{0}\NormalTok{, }\FloatTok{0.8}\NormalTok{, }\FloatTok{0.68}\NormalTok{, }\DecValTok{1}\NormalTok{), }\DataTypeTok{new =} \OtherTok{TRUE}\NormalTok{)}
\CommentTok{# hist(crimtab_long_df[, "height"], axes = FALSE, ann = FALSE)}
\KeywordTok{hist}\NormalTok{(crimtab_long_df}\OperatorTok{$}\NormalTok{height, }\DataTypeTok{axes =} \OtherTok{FALSE}\NormalTok{, }\DataTypeTok{ann =} \OtherTok{FALSE}\NormalTok{)}
\KeywordTok{par}\NormalTok{(}\DataTypeTok{fig =} \KeywordTok{c}\NormalTok{(}\FloatTok{0.68}\NormalTok{, }\DecValTok{1}\NormalTok{, }\DecValTok{0}\NormalTok{, }\FloatTok{0.8}\NormalTok{), }\DataTypeTok{new =} \OtherTok{TRUE}\NormalTok{)}
\CommentTok{# barplot(table(cut(crimtab_long_df[, "finger"], breaks = 10)), space = 0, col = "white", horiz = TRUE, axes = FALSE, axisnames = FALSE)}
\KeywordTok{barplot}\NormalTok{(}\KeywordTok{table}\NormalTok{(}\KeywordTok{cut}\NormalTok{(crimtab_long_df}\OperatorTok{$}\NormalTok{finger, }\DataTypeTok{breaks =} \DecValTok{10}\NormalTok{)), }
        \DataTypeTok{space =} \DecValTok{0}\NormalTok{, }\DataTypeTok{col =} \StringTok{"white"}\NormalTok{, }\DataTypeTok{horiz =} \OtherTok{TRUE}\NormalTok{, }\DataTypeTok{axes =} \OtherTok{FALSE}\NormalTok{, }\DataTypeTok{axisnames =} \OtherTok{FALSE}\NormalTok{)}
\end{Highlighting}
\end{Shaded}

\includegraphics{crimtab_graphics_files/figure-latex/plot with marginals-1.pdf}

\begin{Shaded}
\begin{Highlighting}[]
\KeywordTok{par}\NormalTok{(}\DataTypeTok{fig =} \KeywordTok{c}\NormalTok{(}\DecValTok{0}\NormalTok{, }\DecValTok{1}\NormalTok{, }\DecValTok{0}\NormalTok{, }\DecValTok{1}\NormalTok{))}
\KeywordTok{par}\NormalTok{(}\DataTypeTok{mar =} \KeywordTok{c}\NormalTok{(}\DecValTok{5}\NormalTok{, }\DecValTok{4}\NormalTok{, }\DecValTok{1}\NormalTok{, }\DecValTok{1}\NormalTok{) }\OperatorTok{+}\StringTok{ }\FloatTok{0.1}\NormalTok{)}
\end{Highlighting}
\end{Shaded}

\hypertarget{persp}{%
\subsubsection{\texorpdfstring{\texttt{persp()}}{persp()}}\label{persp}}

\texttt{persp()}를 활용하면 다양한 각도에서 3차원 겨냥도를 그려볼 수
있음. \(x\) 축은 행, \(y\) 축은 열에 펼쳐진 격자를 0에서 1까지로 조정.
\texttt{theta}와 \texttt{phi}는 박스를 돌려보는 각도이고,
\texttt{expand}는 박스 높이의 상대적인 비율임. \(x\) 축과 \(y\) 축의
라벨 이외에는 디폴트값을 적용시킨 겨냥도와 적절히 조정한 겨냥도를 비교해
볼 것,

\begin{Shaded}
\begin{Highlighting}[]
\KeywordTok{par}\NormalTok{(}\DataTypeTok{mfrow =} \KeywordTok{c}\NormalTok{(}\DecValTok{2}\NormalTok{, }\DecValTok{2}\NormalTok{))}
\KeywordTok{persp}\NormalTok{(crimtab_}\DecValTok{2}\NormalTok{, }
      \DataTypeTok{xlab =} \StringTok{"Finger Length"}\NormalTok{, }
      \DataTypeTok{ylab =} \StringTok{"Height"}\NormalTok{)}
\KeywordTok{persp}\NormalTok{(crimtab_}\DecValTok{2}\NormalTok{, }
      \DataTypeTok{xlab =} \StringTok{"Finger Length"}\NormalTok{, }
      \DataTypeTok{ylab =} \StringTok{"Height"}\NormalTok{, }
      \DataTypeTok{theta =} \DecValTok{90}\NormalTok{, }
      \DataTypeTok{phi =} \DecValTok{30}\NormalTok{, }
      \DataTypeTok{expand =} \FloatTok{0.5}\NormalTok{, }
      \DataTypeTok{scale =} \OtherTok{TRUE}\NormalTok{)}
\KeywordTok{persp}\NormalTok{(crimtab_}\DecValTok{2}\NormalTok{, }
      \DataTypeTok{xlab =} \StringTok{"Finger Length"}\NormalTok{, }
      \DataTypeTok{ylab =} \StringTok{"Height"}\NormalTok{, }
      \DataTypeTok{theta =} \DecValTok{135}\NormalTok{, }
      \DataTypeTok{phi =} \DecValTok{30}\NormalTok{, }
      \DataTypeTok{expand =} \FloatTok{0.5}\NormalTok{, }
      \DataTypeTok{scale =} \OtherTok{TRUE}\NormalTok{)}
\KeywordTok{persp}\NormalTok{(crimtab_}\DecValTok{2}\NormalTok{, }
      \DataTypeTok{xlab =} \StringTok{"Finger Length"}\NormalTok{, }
      \DataTypeTok{ylab =} \StringTok{"Height"}\NormalTok{, }
      \DataTypeTok{theta =} \DecValTok{45}\NormalTok{, }
      \DataTypeTok{phi =} \DecValTok{45}\NormalTok{, }
      \DataTypeTok{expand =} \FloatTok{0.5}\NormalTok{, }
      \DataTypeTok{scale =} \OtherTok{TRUE}\NormalTok{)}
\end{Highlighting}
\end{Shaded}

\includegraphics{crimtab_graphics_files/figure-latex/perspective plot-1.pdf}

\begin{Shaded}
\begin{Highlighting}[]
\KeywordTok{par}\NormalTok{(}\DataTypeTok{mfrow =} \KeywordTok{c}\NormalTok{(}\DecValTok{1}\NormalTok{, }\DecValTok{1}\NormalTok{))}
\end{Highlighting}
\end{Shaded}

\hypertarget{comments}{%
\subsection{Comments}\label{comments}}

Student(W.S. Gosset)이 수행한 바 있는 \ldots{}

\end{document}
